\section{Results}
To measure the quality of classifiers we used the 10-fold corss-validation method. 
We used two datasets.
The first one being a set of labeled sequences labeled positive or negative taken from the VISTA database.
The second dataset used the same positive data as the first one but used random sequences from the human genome as negative data.
The results are detailed in this section. The confusion matrices are show in the appendix.

\subsection{4-mers}
Using the 4-mers as features, we obtained the following results:
\begin{table}[ht]
\centering
\begin{tabular}{|c|c|c|}
  \hline
  \textbf{Metric} & \textbf{Value for first dataset} & \textbf{Value for the second dataset} \\
  \hline
AUC-ROC   & 0.6459 & 0.9918 \\
Accuracy  & 0.6083 & 0.9637 \\
Precision & 0.6062 & 0.9799 \\
Recall    & 0.7630 & 0.9517 \\
F1-Score  & 0.6756 & 0.9656 \\
  \hline
\end{tabular}
\caption{Performance Metrics for 4-mers.}
\end{table}

\subsection{5-mers}
Using the 5-mers as features, we obtained the following results:
\begin{table}[ht]
\centering
\begin{tabular}{|c|c|c|}
  \hline
  \textbf{Metric} & \textbf{Value for first dataset} & \textbf{Value for the second dataset} \\
  \hline
AUC-ROC   & 0.6662 & 0.9915 \\
Accuracy  & 0.6218 & 0.9603 \\
Precision & 0.6243 & 0.9729 \\
Recall    & 0.7348 & 0.9522 \\
F1-Score  & 0.6751 & 0.9624 \\
  \hline
\end{tabular}
\caption{Performance Metrics for 5-mers.}
\end{table}

\subsection{6-mers}
Using the 6-mers as features, we obtained the following results:
\begin{table}[ht]
\centering
\begin{tabular}{|c|c|c|}
  \hline
  \textbf{Metric} & \textbf{Value for first dataset} & \textbf{Value for the second dataset} \\
  \hline
AUC-ROC   & 0.6752 & 0.9896 \\
Accuracy  & 0.6288 & 0.9519 \\
Precision & 0.6351 & 0.9584 \\
Recall    & 0.7183 & 0.9513 \\
F1-Score  & 0.6741 & 0.9548 \\
  \hline
\end{tabular}
\caption{Performance Metrics for 6-mers.}
\end{table}

\subsection{Comparison}
The results clearly show two things. 
First the larger the k-mer the better the results. 
Although it gives better results it also increases the processing time as the number of features grows exponentially with the k growing. 
The calculation for 6-mers took 10 times longer than for 4-mers.

The second thing the results show it that the second dataset gives much better results than the first one.
The accuracy for the second dataset is around 30\% points higher.
This could not be explained due to the quality of the data.
The most likely explanation is that the model realized that the negative data from the second dataset came from a different source (human DNA).


